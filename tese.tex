%%% Exemplo de utilização da classe ITA
%%%
%%%   por        Fábio Fagundes Silveira   -  ffs [at] ita [dot] br
%%%              Benedito C. O. Maciel     -  bcmaciel [at] ita [dot] br
%%%              Giovani Volnei Meinertz   -  giovani [at] ita [dot] br
%%%    	         Hudson Alberto Bode       -  bode [at] ita [dot]br
%%%    	         P. I. Braga de Queiroz    -  pi [at] ita [dot] br
%%%    	         Jorge A. B. Gripp         -  gripp [at] ita [dot] br
%%%    	         Juliano Monte-Mor         -  jamontemor [at] yahoo [dot] com [dot] br
%%%    	         Tarcisio A. B. Gripp      -  tarcisio.gripp [at] gmail [dot] com
%%%
%%%   Versão para overleaf:
%%%   por        Alejandro A. Rios Cruz    - aarc.88@gmail.com
%%%              Saulo Gómez               - sagomezs@unal.edu.co
%%%              Ocimar Santos             - ocimar.acad@gmail.com
%%%
%%%   Template disponibilizado em:
%%%              Overleaf: https://pt.overleaf.com/latex/templates/thesis-template-aeronautics-institute-of-technology-ita/yhfrqqydpygk
%%%
%%%   Contribuia você também!
%%%              GitHub:   https://github.com/AlejandroRios/Template_Thesis_ITA
%%%
%%%  IMPORTANTE: O texto contido neste exemplo nao significa absolutamente nada.  :-)
%%%              O intuito aqui eh demonstrar os comandos criados na classe e suas
%%%              respectivas utilizacoes.
%%%
%%%  Tese.tex  2016-08-25
%%%  $HeadURL: http://www.apgita.org.br/apgita/teses-e-latex.php $
%%%
%%% ITALUS
%%% Instituto Tecnológico de Aeronáutica --- ITA, Sao Jose dos Campos, Brasil
%%%                   http://groups.yahoo.com/group/italus/
%%% Discussion list: italus {at} yahoogroups.com
%%%
%++++++++++++++++++++++++++++++++++++++++++++++++++++++++++++++++++++++++++++++
% Para alterar o TIPO DE DOCUMENTO, preencher a linha abaixo \documentclass[?]{?}
%   \documentclass[tg]{ita}			= Trabalho de Graduacao
%   \documentclass[tgfem]{ita}	= Para Engenheiras
%   								msc     		= Dissertacao de Mestrado
%   								mscfem   		= Para Mestras
%   								dsc      		= Tese de Doutorado
%   								dscfem   		= Para Doutoras
%   								quali    		= Exame de Qualificacao
%   								qualifem 		= Exame de Qualificacao para Doutoras
% Para 'Draft Version'/'Versao Preliminar' com data no rodape, adicionar 'dv':
%   \documentclass[dsc, dv]{ita}
% Para trabalhos em Inglês, adicionar 'eng':
%   \documentclass[dsc, eng]{ita}
%		\documentclass[dsc, eng, dv]{ita}
%++++++++++++++++++++++++++++++++++++++++++++++++++++++++++++++++++++++++++++++
\documentclass[dsc]{ita}    % ITA.cls based on standard book.cls
% Quando alterar a classe, por exemplo de [msc] para [msc, eng]) rode mais uma vez o botão BUILD OUTPUT caso haja erro
\usepackage{ae}
\usepackage{graphicx}
\usepackage{epsfig}
\usepackage{amsmath}
\usepackage{amssymb}
\usepackage{subfig}
\usepackage{multirow}
\usepackage{float}
\usepackage{amsthm}
\usepackage{url}         % formats URL addresses properly
\usepackage{appendix}    % allows appendix section to be included
\usepackage{lscape}      % allows a page to be rendered in landscape mode
\usepackage{multicol}    % allows text in multi columns
\usepackage{cancel}      % needed to show canceled terms in equations
\usepackage{lettrine}
\usepackage{float}
\usepackage{placeins}


%HHHHHHHHHHHHHHHHHHHHHHHHHHHHHHHHHHHHHHHHHHHHHHHHHHHHHHHHHHHHHHHHHHHHHHHHHHHHHHHHHHHHHHHHHHHHHHHHHHHHHHHHHHHH
\addbibresource{Referencias/referencias.bib}

%HHHHHHHHHHHHHHHHHHHHHHHHHHHHHHHHHHHHHHHHHHHHHHHHHHHHHHHHHHHHHHHHHHHHHHHHHHHHHHHHHHHHHHHHHHHHHHHHHHHHHHHHHHHH
%\usepackage{subfigure}
%\usepackage{subfigmat}
%PACOTEFIGURAS_SE _ERRADO_ESXCLUIR_ACIMA
\usepackage{booktabs}
%PACOTETABELAS_SE _ERRADO_ESXCLUIR_ACIMA
%HHHHHHHHHHHHHHHHHHHHHHHHHHHHHHHHHHHHHHHHHHHHHHHHHHHHHHHHHHHHHHHHHHHHHHHHHHHHHHHHHHHHHHHHHHHHHHHHHHHHHHHHHHHH

%++++++++++++++++++++++++++++++++++++++++++++++++++++++++++++++++++++++++++++++
% Espaçamento padrão de todo o documento
%++++++++++++++++++++++++++++++++++++++++++++++++++++++++++++++++++++++++++++++
\onehalfspacing

%singlespacing Para um espaçamento simples
%onehalfspacing Para um espaçamento de 1,5
%doublespacing Para um espaçamento duplo

%++++++++++++++++++++++++++++++++++++++++++++++++++++++++++++++++++++++++++++++
% Identificacoes (se o trabalho for em inglês, insira os dados em inglês)
% Para entradas abreviadas de Professora (Profa.) em português escreva: Prof$^\textnormal{a}$.
%++++++++++++++++++++++++++++++++++++++++++++++++++++++++++++++++++++++++++++++
\course{Engenheria da Computação}  % Programa de PG ou Curso de Graduação
%\area{Aircraft Design} % Área de concentração na PG (Não utilizado no caso de TG)

% Autor do trabalho: Nome Sobrenome
\authorgender{fem}                     %sexo: masc ou fem
\author{Maria das Graças}{Silva}
\itaauthoraddress{Rua H8X, Ap. XXX}{12.228-46?}{São José dos Campos--SP}

% Titulo da Tese/Dissertação
\title{Uma Abordagem Sobre o Dilema do Cupim Frente ao Concreto Armado Utilizando Diferentes Composições Cimentísticas}

% Orientador
\advisorgender{masc}                    % masc ou fem
\advisor{Prof.~Dr.}{Adalberto Santos Dupont}{ITA}

% Coorientador (Caso não haja coorientador, colocar ambas as variáveis \coadvisorgender e \coadvisor comentadas, com um % na frente)
\coadvisorgender{fem}									% masc ou fem
\coadvisor{Prof$^\textnormal{a}$.~Dr$^\textnormal{a}$.}{Doralice Serra}{OVNI}

% Pró-reitor da Pós-graduação
\bossgender{masc}												% masc ou fem
\boss{Prof.~Dr.}{John von Neumann}

%Coordenador do curso no caso de TG
\bosscoursegender{masc}									% masc ou fem
\bosscourse{Prof.~Dr.}{John Walker}

% Palavras-Chaves informadas pela Biblioteca -> utilizada na CIP
\kwcip{Cupim}
\kwcip{Dilema}
\kwcip{Construção}

% membros da banca examinadora

\examiner{Prof. Dr.}{Alan Turing}{Presidente}{ITA}
\examiner{Prof. Dr.}{Linus Torwald}{}{UXXX}
\examiner{Prof. Dr.}{Richard Stallman}{}{UYYY}
\examiner{Prof. Dr.}{Donald Duck}{}{DYSNEY}
\examiner{Prof. Dr.}{Mickey Mouse}{}{DISNEY}

% Data da defesa (mês em maiúsculo, se trabalho em inglês, e minúsculo se trabalho em português)
\date{23}{novembro}{2022}

% Número CDU - (somente para TG)
\cdu{???.??}

% Glossario
\makenoidxglossaries % to use glossaries pkg
\frontmatter


%%%%%%%% acronyms %%%%%%%%
\newacronym{ctq}{CTq}{computed torque}
\newacronym{dc}{DC}{direct current}
\newacronym{ear}{EAR}{Equação Algébrica de Riccati}
\newacronym{graus_liberdade}{GDL}{graus de liberdade}
\newacronym{isr}{ISR}{interrupção de serviço e rotina}
\newacronym{lasi}{LASI}{Laboratório	de sistemas inteligentes}


\newglossaryentry{comprimento}{name=\ensuremath{\alpha}, description={Comprimento}, sort={a}}

\newglossaryentry{altura}{name=\ensuremath{\beta}, description={Altura}, sort={b}}

\newglossaryentry{velocidade}{name=\ensuremath{v}, description={Velocidade}, sort={v}}

\newglossaryentry{vet_distancia}{name=\ensuremath{\textbf{a}}, description={Vetor de distâncias}, sort={a_bold}}

\newglossaryentry{vet_unitario}{name=\ensuremath{\textbf{e}_{j}}, description={Vetor unitário de dimensão $n$ e com o $j$-ésimo componente igual a $1$}, sort={e_j}}

\newglossaryentry{delta_kro}{name=\ensuremath{\delta_{k-k_f}}, description={Delta de Kronecker no instante $k_f$}, sort={d_k}}



\begin{document}
% Folha de Rosto e Capa para o caso do TG
\maketitle

% Dedicatoria: Nao esqueca essa secao  ... :-)
\begin{itadedication}
Aos amigos da Graduação e Pós-Graduação do ITA por motivarem tanto a criação deste template pelo Fábio Fagundes Silveira quanto por motivarem a mim e outras pessoas a atualizarem e aprimorarem este excelente trabalho.
\end{itadedication}

% Agradecimentos
\begin{itathanks}
\input{PreTextuais/agradecimentos}
\end{itathanks}

% Epígrafe
\thispagestyle{empty}
\ifhyperref\pdfbookmark[0]{\nameepigraphe}{epigrafe}\fi
\begin{flushright}
\begin{spacing}{1}
\mbox{}\vfill
\input{PreTextuais/epigrafe}
\end{spacing}
\end{flushright}

% Resumo
\begin{abstract}
\noindent
\input{PreTextuais/resumo}
\end{abstract}

% Abstract
\begin{englishabstract}
\noindent
\input{PreTextuais/abstract}
\end{englishabstract}

% Lista de figuras
\listoffigures %opcional

% Lista de tabelas
\listoftables %opcional

% Lista de abreviaturas
\printnoidxglossary[type=\acronymtype,
    title=\listofabbreviationsname,
    toctitle=\listofabbreviationsname,
    sort=standard,
    nonumberlist]
    
% Lista de simbolos
\printnoidxglossary[
    title=\listofsymbolsname,
    toctitle=\listofsymbolsname,
    sort=use,   % according to abnt 14724 the list of symbols is sorted by order of use in text
    nonumberlist]

% Sumario
\tableofcontents


\mainmatter
% Os capitulos comecam aqui

\chapter{Introdução}
\section{Objetivo}
O objetivo deste projeto de mestrado é desenvolver técnicas de controle subótimo das juntas passivas (não atuadas) de um robô subatuado, incluindo o estudo teórico do tema, proposição de um método de controle e sua verificação
experimental em um manipulador de três graus de liberdade \cite{Nascimento1970}.

O teste \cite{Patagonios2001} e validação das técnicas de controle propostas foram realizados em um ambiente de simulação e no manipulador
experimental, adquirido através do projeto FAPESP $N^{\circ}$ 98/00649-5, que se encontra em funcionamento no \gls{lasi} do Departamento de Engenharia Elétrica da USP em São Carlos. De acordo com \textcite{Furmento1995}, pode-se listar:
\begin{itemize}
\item Isso;
\item Aquilo; e
\item Aquele outro.
\end{itemize}

Então, no \gls{lasi} são realizadas as várias atividades listadas.

\section{Motivação}
Manipuladores mecânicos \cite{Sbornian2002} vêm sendo utilizados há várias décadas para a automação de tarefas
repetitivas em ambientes industriais, ambientes estes de fácil acesso tanto em termos físicos quanto em termos de baixo
risco à saúde humana. Nos últimos anos, verifica-se uma utilização cada vez maior de manipuladores em
ambientes de difícil acesso ou inóspitos, como no interior de usinas nucleares, no fundo dos oceanos e no
espaço. A localização dos manipuladores nesta nova gama de aplicações faz com que sua manutenção,após uma falha mecânica ou elétrica, seja custosa e demorada, portanto estes mecanismos requerem sofisticadas
metodologias de controle tolerante a falhas \cite{ITALUS2004}.

Após a ocorrência de uma falha em um de seus atuadores, o manipulador torna-se um sistema subatuado. Um sistema também pode se tornar subatuado quando é projetado  dessa maneira, ou quando o operador deliberadamente mantém um ou mais atuadores disponíveis inoperantes durante uma tarefa. Reduzindo o número de atuadores sem reduzir o número de graus de
liberdade e ajustando-se o sistema de controle adequado, pode-se obter um mecanismo cujo consumo de energia é menor, mas cujas propriedades são mantidas \cite{Arystides1994}.

\begin{figure}[ht]
\centering
\includegraphics[width=0.5\textwidth]{Cap1/cupim}
\caption{Proibido estacionar cupins. Legenda grande, com o objetivo de demonstrar a indentação na lista de figuras.}
\label{cupim}
\end{figure}

Controle do manipulador após uma falha é fundamental do ponto de vista de operação, principalmente nos casos descritos acima, em que a localização do manipulador impede sua manutenção de forma fácil. Recentemente tem havido a combinação
de algorítmos de detecção e isolação de falhas com os de controle pós-falha em um método unificado. Uma extensão desse trabalho, que vê o problema de controle tolerante a falhas através de uma perspectiva integrada, foi proposta por
{marcel4}. Os autores apresentam um ambiente híbrido consistindo de três unidades básicas que garantem a compleição de tarefas na presença de qualquer número de juntas falhas (Fig.~\ref{cupim}). A primeira unidade é um esquema de detecção
e isolação de falhas que continuamente monitora o manipulador para detectar e identificar possíveis falhas nas juntas. A segunda unidade é responsável pela reconfiguração do controle. A terceira unidade é composta de algorítmos de
controle apropriados para cada tipo de configuração do robô, baseado na informação da unidade de reconfiguração \cite{COFFEE2000}.

No presente trabalho nos concentramos na unidade de algorítmo de controle, e mais especificamente no problema de controle da posição  angular de uma junta falha para qualquer posição desejada de uma maneira subótima, quando dispomos
de redundância de atuação para a realização dessa tarefa. O termo subótimo se deve ao fato de que não há garantias de otimalidade em vista das não-linearidades inerentes ao sistema e de outros fatores que serão abordados nos capítulos posteriores. Ao longo do texto, para simplificação, usaremos tanto o termo subótimo como ótimo para nos referirmos à metodologia utilizada.

Segundo, o critério de otimização utilizado será o acoplamento entre as juntas do
manipulador e neste caso, temos um sistema redundante quando ocorre falha de uma das juntas do manipulador de três juntas, e seu posicionamento é controlado pelas duas restantes. Nossa solução para o problema é baseada na formulação
de redundância local, extensivamente estudada no contexto de cinemática inversa ({nakamura}). A principal contribuição deste trabalho é a extensão deste método usando as equações dinâmicas de manipuladores subatuados e a utilização do índice de acoplamento como um critério para a minimização do torque e da energia gasta pelo sistema durante o controle das juntas falhas.

\begin{figure}[ht!]
\centering
\includegraphics[width=1\textwidth]{Cap1/cupimconcreto}
\caption{Exemplo real de cupim frente ao seu dilema.}
\label{FDII}
\end{figure}

\section{Organização do trabalho}
\subsection{Sub-organização}
O capítulo 1 contém a introdução do trabalho, onde são expostos o objetivo, a motivação do mesmo, a descrição do sistema e a formulação do problema com a nomenclatura utilizada; além de uma revisão bibliográfica da literatura relacionada ao tema do trabalho.

\subsubsection{SubSub-organização}

No capítulo 2 apresentamos a modelagem dinâmica de um manipulador subatuado e o conceito de índice de acoplamento para medir o acoplamento dinâmico entre as juntas ativas e passivas. Este índice é utilizado para a análise e projeto de uma metodologia de controle subótimo do manipulador.

\subsubsection{Outra subsub-organizacao}

O capítulo 3 apresenta o controle subótimo de manipuladores através de redundância de atuação. Descreve-se a técnica de controle ponto a ponto de manipuladores subatuados. A seguir mostramos  a linearização destes por realimentação, cujo efeito é linearizar e desacoplar o sistema não linear. Finalmente é proposta uma sequência de controle subótimo local das juntas passivas visando a minimização de certos critérios como torque, velocidade e em particular a energia consumida pelo sistema. Este é de fato o tema principal deste mestrado.

É também apresentado no capítulo 4 um resumo do projeto de controladores  $H_{2}$ e $H_{\infty}$, cuja principal vantagem é a robustez na presença de incertezas paramétricas e distúrbios externos.

O capítulo 5 mostra as características e a operação do robô e do ambiente de simulação utilizados nos testes e experimentação da metodologia apresentada.

Os procedimentos da metodologia e os resultados obtidos para algumas configurações e diferentes controladores encontram-se no capítulo 6.

No capítulo 7 são apresentadas as conclusões do trabalho.

Quatro apêndices fazem parte do trabalho. O apêndice A apresenta alguns tópicos de álgebra linear que são a base do método proposto. No apêndice B são mostradas as equações da matriz de inércia e do vetor de torques não-inerciais
utilizados na modelagem dinâmica do manipulador. No apêndice C temos as expressões literais dessas equações feitas no software MAPLE e no apêndice D alguns programas feitos no software MATLAB utilizados no projeto \cite{Furmento1995, Morgado2003}.


\subsection{Como utilizar o glossário}
O glossário no \LaTeX é automatico, utilizando o pacote ``glossaries'', então basta adicionar as entradas em ``listaabreviaturas.tex'' e em ``listasimbolos.tex'' e utilizar o comando ``$\backslash$gls\{\}''.
A lista de abreviaturas e de simbolos será gerada automaticamente de acordo com os elementos que foram utilizados no texto.

A primeira vez que uma abreviatura é chamada é automaticamente representada em sua forma completa: \gls{ctq}.
Ao utilizar o comando depois da primeira vez, apenas a abreviatura será utilizada: \gls{ctq}.

O glossário também pode ser utilizado para simbolos matemáticos, então pode-se definir por exemplo uma variável delta de Kronecker e utilizar o comando ``$\backslash$gls\{\}'', \gls{delta_kro}. 
Se houver a necessidade de modificar o simbolo utilizado, não precisa alterar todas as entradas ao longo da tese, basta alterar o simbolo em ``listadesimbolos.tex''.




\chapter{Modelagem Dinâmica de Cupins Cibernéticos}
\section{Modelagem no espaço das juntas}
Manipuladores subatuados diferem dos totalmente atuados pois são equipados com um número de atuadores que é sempre menor que o número de \gls{graus_liberdade}. Portanto, nem todos os \gls{graus_liberdade} podem ser controlados ativamente ao mesmo tempo \cite{Sbornian2004}. Por exemplo, com um manipulador planar de 3 juntas equipado com dois atuadores, ou seja, duas juntas ativas e
uma passiva, pode-se controlar ao mesmo tempo duas das juntas a qualquer instante, mas não todas. Para controlar todas as juntas de um manipulador subatuado, deve-se usar um controle sequencial. Este princípio foi provado pela primeira vez por {arai} usando  argumentos dinâmicos linearizados \cite{Joea2003}, e é a base para a modelagem no espaço das juntas e no espaço Cartesiano. A Tabela \ref{minhatab} apresenta os resultados \cite{Assenmacher1993,Silberschatz1991,Caromel1998}.

\begin{table}
\caption{Exemplo de uma Tabela}
\label{minhatab}

\center
\begin{tabular}{cccc}
  % after \\: \hline or \cline{col1-col2} \cline{col3-col4} ...
  \hline
	Parâmetro & Unidade & Valor da simulação & Valor experimental   \\
	\hline
  Comprimento, \gls{comprimento} & $m$ &  $8,23$  & $8,54$ \\
  Altura, \gls{altura} & $m$     &  $29,1$ & $28,3$\\
	Velocidade, \gls{velocidade} & $m/s$  &  $60,2$ & $67,3$\\
	\hline
\end{tabular}
\end{table}

Devido ao fato de que no máximo $n_{a}$ coordenadas generalizadas (ângulos das juntas ou variáveis cartesianas) podem ser controladas num dado instante, o vetor de coordenadas generalizadas é dividido em duas partes, representando as coordenadas generalizadas ativas e as coordenadas generalizadas passivas \cite{Callaghan1995}.

\begin{figure}[ht]
\centering
\includegraphics[width=0.75\textwidth]{Cap2/spiderrobot}
\caption{Cupim cibernético.}\label{FDIII}
\end{figure}

Considerando um robô manipulador rígido, malha aberta, e de $n$-juntas em série. Seja $q$ a representação de seu vetor de posição angular das juntas  e $\tau$ a representação de seu vetor de torque. A equação dinâmica pelo método de
Lagrange é dada por:
\begin{equation} \label{eq:lagr1}
\frac{d}{dt} \left( \frac{\partial L}{\partial \dot{q}} \right) -\frac{\partial L}{\partial q}=\tau^{T}.
\end{equation}
O Lagrangiano $L$ é definido como a diferença entre as energias cinética e potencial do sistema:
\begin{equation} \label{L}
L=T-P
\end{equation}

A energia cinética total dos ligamentos é representada:
\begin{equation} \label{energT}
T=\frac{1}{2}\dot{q}^{T}M(q)\dot{q}
\end{equation}


\chapter{Controle Robusto de Concretos Caóticos}
\input{Cap3/cap3}

\chapter{Conclusão}
\input{Cap4/cap4}

% REFERENCIAS BIBLIOGRAFICAS
\renewcommand\bibname{\itareferencesnamebabel} %renomear título do capítulo referências
\printbibliography

% Apendices
\appendix
\chapter{Tópicos de Dilema Linear} %opcional
\input{ApeA/apendiceA}

% Anexos
\annex
\chapter{Exemplo de um Primeiro Anexo} %opcional
\input{AneA/anexoA}

% Glossario
%\itaglossary
%\printglossary

% Folha de Registro do Documento
% Valores dos campos do formulario
\FRDitadata{25 de março de 2015}
\FRDitadocnro{DCTA/ITA/DM-018/2015} %(o número de registro você solicita a biblioteca)
\FRDitaorgaointerno{Instituto Tecnológico de Aeronáutica -- ITA}
%Exemplo no caso de pós-graduação: Instituto Tecnol{\'o}gico de Aeron{\'a}utica -- ITA
\FRDitapalavrasautor{Cupim; Cimento; Estruturas}
\FRDitapalavrasresult{Cupim; Dilema; Construção}
%Exemplo no caso de graduação (TG):
%\FRDitapalavraapresentacao{Trabalho de Graduação, ITA, São José dos Campos, 2015. \NumPenultimaPagina\ páginas.}
%Exemplo no caso de pós-graduação (msc, dsc):
\FRDitapalavraapresentacao{ITA, São José dos Campos. Curso de Mestrado. Programa de Pós-Graduação em Engenharia Aeronáutica e Mecânica. Área de Sistemas Aeroespaciais e Mecatrônica. Orientador: Prof.~Dr. Adalberto Santos Dupont. Coorientadora: Prof$^\textnormal{a}$.~Dr$^\textnormal{a}$. Doralice Serra. Defesa em 05/03/2015. Publicada em 25/03/2015.}
\FRDitaresumo{\input{PreTextuais/resumo}}
%  Primeiro Parametro: Nacional ou Internacional -- N/I
%  Segundo parametro: Ostensivo, Reservado, Confidencial ou Secreto -- O/R/C/S
\FRDitaOpcoes{N}{O}
% Cria o formulario
\itaFRD

\end{document}
% Fim do Documento. O massacre acabou!!! :-)
